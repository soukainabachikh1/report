\chapter*{Introduction}
\addcontentsline{toc}{chapter}{Introdcution}
\pagenumbering{arabic}

% Dans un monde où la compétitivité économique et la recherche des nouveaux talents
% sont en pleine croissance; présentant, de ce fait, un enjeu et une priorité majeurs pour la
% majorité des entreprises. Cette capacité à cibler, attirer et recruter les meilleurs profils qui
% s’alignent avec les besoins internes est désormais une priorité stratégique pour assurer le
% développement à long terme des organisations. Néanmoins, les méthodes traditionnelles de
% recrutement se heurtent à une myriade de défis. En effet, ces dernières nécessitent souvent
% des efforts immenses de la part des recruteurs pour examiner manuellement un grand
% nombre de candidatures, ce qui les rend alors sujettes à des erreurs humaines, des retards
% ou à des inexactitudes dans l’évaluation des profils. En outre, la lenteur des échanges entre
% les différentes parties prenantes et les problèmes de communication ajoutent des obstacles
% supplémentaires à un processus déjà complexe.
Dans un monde où la compétitivité économique et la recherche 
de nouveaux talents sont en pleine croissance, les 
entreprises doivent relever un défi stratégique majeur : 
attirer et recruter les meilleurs profils qui répondent à 
leurs besoins internes. Cependant, les méthodes 
traditionnelles de recrutement font face à une série de 
difficultés. Elles nécessitent souvent des efforts 
considérables de la part des recruteurs pour examiner 
manuellement un grand nombre de candidatures, ce qui expose 
à des risques d'erreurs, de retards et d'imprécisions dans 
l'évaluation des compétences. 
\newline

Face à ces défis, la digitalisation du processus de recrutement émerge comme une
solution prometteuse, offrant d’une part aux candidats la possibilité de consulter toutes les
offres disponibles, postuler pour celles-ci et suivre l’état d’avancement de sa candidature.
D’une autre part, cette solution permet aux recruteurs de suivre les candidatures dans un endroit centralisé
comporenant toutes les informations relatives aux candidats et d’optimiser les ressources humaines et financières en réduisant les coûts
liés à la gestion des candidatures et en accélérant les délais de recrutement. C’est dans
ce contexte ou s’inscrit mon projet de fin d’études ayant pour objectif primordial de développer une plateforme dédiée à la centralisation
du processus de recrutement au sein de 4D Logiciels, qui répondra à ces
problématiques en améliorant l’efficacité du processus actuel de recrutement.  
\newline

Afin de présenter le travail réalisé, ce rapport est 
structuré en quatre parties principales. Le premier 
chapitre met en lumière l’organisme d’accueil, 4D Logiciels 
Maroc, ainsi que le contexte général du projet et 
la démarche suivie durant le stage. Le deuxième chapitre 
offre une vue d'ensemble sur l'existant et la spécification 
des besoins. Le troisième chapitre se concentre sur la 
modélisation de la solution. Enfin, le dernier chapitre 
présente les outils et les technologies utilisés et présente la réalisation 
du projet en mettant en exerque les interfaces développées et la validation 
de la plateforme.
% les différentes interfaces développées avec les tests effectués.

% Le recrutement est crucial pour toute entreprise, influant directement sur sa performance et sa réussite dans un marché du travail compétitif en constante évolution. Ce rapport se concentre sur la digitalisation des processus de recrutement chez 4D Logiciels Maroc, dans le cadre de mon stage de fin d'études à l'INPT. Le projet vise à développer une plateforme innovante pour simplifier et automatiser les différentes étapes du recrutement, répondant ainsi aux besoins contemporains du marché du travail.
% \newline

% Ce projet s'articule autour de plusieurs axes majeurs. Tout d'abord, il cherche à centraliser et optimiser les processus de recrutement en intégrant des fonctionnalités avancées telles que la gestion des offres d'emploi, le tri automatique des CVs et la planification des entretiens. Ensuite, il s'appuie sur les meilleures pratiques en ingénierie logicielle pour garantir la solidité et l'efficacité de la solution. Enfin, il adopte une démarche agile, favorisant des itérations rapides et une adaptation continue pour répondre aux besoins changeants de l'entreprise et du marché.
% \newline

% Le projet sera structuré en plusieurs chapitres. Nous commencerons par présenter le contexte général du projet, suivi par une analyse et une spécification des besoins de l'entreprise en matière de recrutement. Ensuite, nous détaillerons la conception de la solution, en mettant en avant les différentes fonctionnalités prévues. Enfin, nous aborderons l'implémentation de la solution et sa validation pour assurer sa conformité aux attentes et son bon fonctionnement.
% \newline
