\chapter*{Introduction}
\addcontentsline{toc}{chapter}{Introdcution}
Dans un monde où la compétitivité économique et la recherche des nouveaux talents
sont en pleine croissance ; présentant, de ce fait, un enjeu et une priorité majeurs pour la
majorité des entreprises. Cette capacité à cibler, attirer et recruter les meilleurs profils qui
s’alignent avec les besoins internes est désormais une priorité stratégique pour assurer le
développement à long terme des organisations. Néanmoins, les méthodes traditionnelles de
recrutement se heurtent à une myriade de défis. En effet, ces dernières nécessitent souvent
des efforts immenses de la part des recruteurs pour examiner manuellement un grand
nombre de candidatures, ce qui les rend alors sujettes à des erreurs humaines, des retards
ou à des inexactitudes dans l’évaluation des profils. En outre, la lenteur des échanges entre
les différentes parties prenantes et les problèmes de communication ajoutent des obstacles
supplémentaires à un processus déjà complexe.
\newline

Face à ces défis, la digitalisation du processus de recrutement émerge comme une
solution prometteuse, offrant d’une part aux candidats la possibilité de consulter toutes les
offres disponibles, postuler pour celles-ci et suivre l’état d’avancement de sa candidature.
D’une autre part, cette solution permet aux recruteurs d’utiliser des outils d’analyse des
données pour évaluer les candidatures, faciliter la communication entre toutes les parties
prenantes, et d’optimiser les ressources humaines et financières en réduisant les coûts
liés à la gestion des candidatures et en accélérant les délais de recrutement. C’est dans
ce contexte ou s’inscrit mon projet de fin d’études ayant pour objectif primordial de
concevoir et de développer une plateforme dédiée à la digitalisation et la centralisation
du processus de recrutement au sein de 4D Logiciels qui répondra bien évidemment à ces
problématiques en améliorant l’efficacité du processus actuel de recrutement.  
\newline

Afin de présenter le travail réalisé, ce rapport est structuré en quatre parties principales. 
Le premier chapitre met en lumière l’organisme d’accueil de mon stage : 4D Logiciels Maroc,
 le contexte général du projet ainsi que la démarche suivie pour le développement
du projet. Le deuxième chapitre donne une vision globale sur l’existant et la spécification
des besoins. Le troisième chapitre est consacré à la modélisation UML et BPMN de la
solution, ainsi, il met en évidence les outils technologiques utilisés. Finalement, le dernier
chapitre présente la réalisation du projet en mettant en exergue l’architecture technique
adoptée et les différentes interfaces réalisées.
