% Define the page style
\fancypagestyle{chapterstyle}{
   \fancyhead[L]{\nouppercase{\rightmark}}
   \fancyhead[R]{Projet de fin d'etudes 2023-2024}
   \fancyfoot[C]{\vspace{20pt}\thepage} % Adjust the vertical space here
   \setlength{\headheight}{20pt}
   \setlength{\footskip}{30pt} % Adjust the value as needed
}


\chapter{Conception de la solution}
\pagestyle{chapterstyle}
Ce chapitre détaille la conception de notre solution en se 
concentrant sur l'architecture technique. Après avoir défini les 
besoins et les cas d'utilisation, nous utiliserons des maquettes 
pour visualiser l'interface utilisateur, des diagrammes de séquence 
pour illustrer les interactions dynamiques entre les composantes 
de l'application et des diagrammes de classes pour structurer les 
éléments et leurs relations. Cette étape assure une mise en œuvre 
cohérente et efficace, intégrant toutes les exigences identifiées.

\newpage
\vspace{1cm}


\section{Maquettes}
Le design UX/UI est un élément crucial de notre projet, 
car il vise à aligner les objectifs du client et ses attentes avec le travail qui va être réalisé par la suite.
\newline
Dans notre situation, nous avons converti les besoins fonctionnels du système en maquettes. Cela nous a permis de préparer l'interface globale avant d'entrer dans la phase de développement. Ensuite, nous avons sollicité les retours des utilisateurs finaux afin d'améliorer encore notre application.
\newline



\begin{figure}[htbp]
   \centering
   \includegraphics[scale=0.8]{Images/1.jpg} % Replace with the actual filename of the IBM logo image
   \caption{Maquettes Authentification et Profile}
   \label{fig:maquette1}
\end{figure}
\vspace{1cm}

\begin{figure}[htbp]
   \centering
   \includegraphics[scale=0.8]{Images/2.jpg} 
   \caption{Maquettes Offres}
   \label{fig:maquette2}
\end{figure}

\begin{figure}[htbp]
   \centering
   \includegraphics[scale=1]{Images/3.jpg} 
   \caption{Maquettes historique de candidature et test}
   \label{fig:maquette3}
\end{figure}

\begin{figure}[htbp]
   \centering
   \includegraphics[scale=1.4]{Images/4.jpg} 
   \caption{Maquettes Dashboard recruteur}
   \label{fig:maquette4}
\end{figure}

\begin{figure}[htbp]
   \centering
   \includegraphics[scale=1.2]{Images/5.jpg} 
   \caption{Maquettes créer une offre et consulter calendrier}
   \label{fig:maquette5}
\end{figure}


\section{Architecture de l'application}
\subsection{Architecture physique}
Nous avons choisi d'adopter une architecture client/serveur 
multi-tiers pour notre application. Ainsi, l'accès à 
l'application nécessite le transit par des requêtes HTTP pour 
récupérer et déposer des données dans le dépôt central. De plus, 
nous avons centralisé la gestion de la base de données du système, 
la séparant de la logique métier pour faciliter la maintenance. 
Enfin, l'application est répartie sur plusieurs serveurs, chacun 
responsable d'une tâche spécifique. Cette répartition des tâches 
entre les serveurs permet d'assurer une grande souplesse, 
des performances optimales et des temps de réponse rapides. 
La figure suivante illustre l'architecture physique que nous avons mise en place :


\begin{figure}[htbp]
   \centering
   \includegraphics[scale=0.6]{Images/physique.jpg} % Replace with the actual filename of the IBM logo image
   \caption{Architecture physique}
   \label{fig:physiqueArch}
\end{figure}

Cette architecture se compose principalement des éléments suivants :
\begin{itemize}
   \item[•] \textbf{Serveur REST :} Un serveur web qui suit les principes de l’architecture REST et expose des ressources via des URI, permettant aux clients d’effectuer des opérations standardisées sur ces ressources pour accéder aux données et fonctionnalités du serveur.
   \item[•] \textbf{Serveur 4D :} Ce serveur contient la couche métier de notre application.
   \item[•] \textbf{Serveur de base de données :} Ce serveur se charge de la gestion du stockage des données.
   \item[•] \textbf{Couche réseau :} Le protocole TLS sécurise les connexions client/serveur en cryptant les données échangées, permettant ainsi de renforcer la sécurité de votre application 4D Server.
\end{itemize}

\subsection{Architecture logique}
Pour avoir une architecture robuste, modulable et évolutive, il faut utiliser le principe de « Couche
», et donc séparer au maximum les différents types de traitement de l’application. L’environnement de travail n’est pas dépendant à une technologie spécifique. Pour cette raison, nous avons utilisé plu- sieurs technologies afin de développer une solution aboutie, performante, multicouches et qui s’intègre parfaitement. La figure suivante illustre l’architecture logicielle proposée pour le système développé, en présentant trois couches : couche présentation (web), couche métier qui s’occupe des différents traite- ments et couche accès aux données.
\begin{figure}[htbp]
   \centering
   \includegraphics[scale=0.1]{Images/logique.png} 
   \caption{Architecture logique}
   \label{fig:logiqueArch}
\end{figure}

Au niveau 4D Server, notre développement s’est concentré principalement sur la couche métier. En effet, 4D Server offre un environnement de développement qui simplifie considérablement la création d’applications. Les autres couches, telles que la couche d’accès aux données et la couche de présentation, sont déjà implémentées et intégrées dans 4D.Ainsi, les développeurs peuvent se concentrer sur la logique métier de leurs applications sans avoir à se soucier des détails techniques des autres couches. Cette approche permet un développement rapide et efficace, tout en offrant des fonctionnalités avancées pour répondre aux besoins spécifiques des projets.
\newline
Aussi, nous avons travaillé avec ORDA, est une technologie spécifique qui facilite l’accès à une base de données relationnelle en tant qu’objets. Elle permet de manipuler les données de la base de données à l’aide d’un langage de programmation orienté objet ou d’interfaces utilisateur spécifiques. ORDA simplifie l’interaction avec la base de données en fournissant des abstractions supplémentaires et en masquant certaines complexités liées aux requêtes SQL.
\newline
ORDA nous permet de créer des fonctions de classe de haut niveau au-dessus du modèle de données. Cela nous permet d’écrire du code orienté métier et de le «publier» comme une API. Le datastore, les
dataclasses, les entity selections et les entités sont tous disponibles en tant qu’objets de classe pouvant contenir des fonctions.

\begin{figure}[htbp]
   \centering
   \includegraphics[scale=0.3]{Images/logique 2.jpg} 
   \caption{Architecture logique}
   \label{fig:logiqueArch}
\end{figure}

Grâce à 4D, les développeurs peuvent se concentrer sur l’essentiel et créer des applications puissantes et performantes en toute simplicité.


\subsection{Architecture technique}

\begin{figure}[htbp]
   \centering
   \includegraphics[scale=0.6]{Images/physique.jpg} % Replace with the actual filename of the IBM logo image
   \caption{Architecture physique}
   \label{fig:seq4}
\end{figure}

Le connecteur s’occupe de la récolte des données saisies par l’utilisateur dans le navigateur, ces données sont envoyées au serveur 4D via des requêtes HTTP. La couche web récupère les données reçues et les transmet à la couche métier qui effectue les traitements nécessaires. La couche 4D Database s’occupe de la sérialisation et la dé-sérialisation.

\section{Conception détaillée}
\subsection{Diagramme de Classes}
Le diagramme de classes est l'un des diagrammes statiques d'UML. 
Il permet de représenter la structure d'un système informatique en 
présentant les différentes classes, leurs attributs, leurs méthodes, 
ainsi que les relations entre elles.

\begin{figure}[htbp]
   \centering
   \includegraphics[scale=0.3]{Images/classDiagram.png} % Replace with the actual filename of the IBM logo image
   \caption{Diagramme de classes}
   \label{fig:ClassDiag}
\end{figure}


\subsection{Diagrammes de séquence}
Le diagramme de séquence offre une représentation claire 
et détaillée des interactions entre les divers objets ou 
composants d’un système. Dans le cadre de notre projet, 
ce diagramme nous permet de modéliser de manière précise et 
compréhensible les échanges entre les utilisateurs et l’interface 
utilisateur de l’application, ainsi que les interactions entre cette 
dernière, le serveur et la base de données associée.

\begin{figure}[htbp]
   \centering
   \includegraphics[scale=0.8]{Images/auth.png} % Replace with the actual filename of the IBM logo image
   \caption{Diagramme de séquence d’authentification}
   \label{fig:seq1}
\end{figure}

\begin{figure}[htbp]
   \centering
   \includegraphics[scale=0.8]{Images/creerOffre.png} % Replace with the actual filename of the IBM logo image
   \caption{Diagramme de séquence de création des offres}
   \label{fig:seq2}
\end{figure}

\begin{figure}[htbp]
   \centering
   \includegraphics[scale=0.6]{Images/postuler.png} % Replace with the actual filename of the IBM logo image
   \caption{Diagramme de séquence de postulation}
   \label{fig:seq3}
\end{figure}

\begin{figure}[htbp]
   \centering
   \includegraphics[scale=0.8]{Images/gererCandidature.png} % Replace with the actual filename of the IBM logo image
   \caption{ Diagramme de séquence de gestion des candidatures}
   \label{fig:seq4}
\end{figure}
