% Define the page style
\fancypagestyle{chapterstyle}{
   \fancyhead[L]{\nouppercase{\rightmark}}
   \fancyhead[R]{Projet de fin d'etudes 2023-2024}
   \fancyfoot[C]{\vspace{20pt}\thepage} % Adjust the vertical space here
   \setlength{\headheight}{20pt}
   \setlength{\footskip}{30pt} % Adjust the value as needed
}


\chapter{Implémentation et Validation}
\pagestyle{chapterstyle}
Dans la suite du projet, l'implémentation et la validation 
sont d'une importance capitale. Le choix des outils de 
développement impacte significativement le temps nécessaire à 
la programmation et la qualité du produit final. Cette phase 
consiste à concrétiser le modèle conceptuel précédemment établi 
en composants logiciels formant notre système, puis à vérifier 
leur bon fonctionnement. Dans ce chapitre, nous allons présenter 
de manière succincte les différents outils que nous avons utilisés 
tout au long du développement et de la validation de notre 
application.
Ainsi le travail réalisé.

\newpage
\vspace{1cm}

\section{Outils et technologies de développement}
\subsection{Outils de conception}

\large 
\textbf{Figma}

\begin{figure}[htbp]
   \centering
   \includegraphics[scale=0.09]{Images/vs.png} 
   \caption{Logo Figma}
   \label{fig:vscode}
\end{figure}
Figma bla bla 

\large 
\textbf{Enterprise Architect}
\begin{figure}[htbp]
   \centering
   \includegraphics[scale=0.09]{Images/vs.png} 
   \caption{Logo Enterprise Architect}
   \label{fig:vscode}
\end{figure}
% \Large
% \textbf{Visual studio}

\subsection{Environnement de developpement}
\large 
\textbf{Visual Studio Code}

\begin{figure}[htbp]
   \centering
   \includegraphics[scale=0.09]{Images/vs.png} 
   \caption{Logo Visual Stduio Code}
   \label{fig:vscode}
\end{figure}

Visual Studio Code est éditeur de texte open source, gratuit et multiplateforme (Windows, Mac et Linux), développé
par Microsoft. Principalement conçu pour le développement
d’application avec JavaScript, Type script et Node.js, l’éditeur peut s’adapter à d’autres types de langages grâce à un
Système d’extension bien fourni.
\newline

\large 
\textbf{4D Client}

\begin{figure}[htbp]
   \centering
   \includegraphics[scale=0.2]{Images/4dcl.png} 
   \caption{Logo 4D Client}
   \label{fig:4dcl}
\end{figure}

4D vous permet de construire des applications client-serveur 
personnalisées qui sont homogènes, multiplateformes et avec 
une option de mise à jour automatique. Les applications 
client et serveur sont configurées dans la page Client/Serveur 
de la boîte de dialogue Construire une application.
\newline

\large 
\textbf{4D Serveur}

\begin{figure}[htbp]
   \centering
   \includegraphics[scale=0.2]{Images/4dsrv.png} 
   \caption{Logo 4D Serveur}
   \label{fig:4dsrv}
\end{figure}

4D Server est un composant logiciel de la plateforme de développement 4D qui permet le déploiement et la gestion
d’applications client-sebrveur. Il offre un environnement robuste et évolutif pour héberger des applications 4D, 
permettant à plusieurs utilisateurs d’y accéder et d’interagir avec
l’application simultanément. 4D Server agit comme un hub centralisé, 
gérant le stockage des données, le traitement et la communication entre 
le serveur et les applications clientes connectées. Il prend en charge 
des fonctionnalités telles que l’accès simultané aux données partagées, 
la gestion des transactions, les contrôles de sécurité et la collaboration
multi-utilisateur.
\newline


\large 
\textbf{Postman}

\begin{figure}[htbp]
   \centering
   \includegraphics[scale=0.4]{Images/postman.png} 
   \caption{Logo postman}
   \label{fig:4dsrv}
\end{figure}

Postman fournit une interface conviviale où les développeurs 
peuvent spécifier les paramètres de requête, les entêtes, 
les corps de requête, les méthodes HTTP, etc. Ils peuvent 
également inspecter les réponses des serveurs, valider les 
résultats et effectuer des tests automatisés pour s’assurer que 
l’API fonctionne correctement. 
\newline

\large 
\textbf{GitLab}

\begin{figure}[htbp]
   \centering
   \includegraphics[scale=0.6]{Images/gitlab.jpg} 
   \caption{Logo GitLab}
   \label{fig:gitlab}
\end{figure}
GitLab est une plateforme DevOps complète proposée sous la forme 
d’une application unique. Elle révolutionne le développement, 
la sécurité, l’exploitation et la collaboration entre les équipes. 
Créez, testez et déployez des logiciels plus rapidement en 
n’utilisant qu’une seule solution. 

\subsection{Langages de programmation}
\large 
\textbf{4D}

\begin{figure}[htbp]
   \centering
   \includegraphics[scale=0.8]{Images/logo-4d.jpg} 
   \caption{Logo 4D}
   \label{fig:4D}
\end{figure}
Le langage 4D est un langage de programmation spécifique à la
 plateforme utilisé dans l’environnement de développement 4D pour
  créer des applications professionnelles et des bases de données. 
  Il est conçu pour simplifier le développement d’applications 
  en fournissant des fonctionnalités spécifiques à la gestion des 
  données et des interfaces utilisateur.
\newline

\large 
\textbf{TypeScript}
\begin{figure}[htbp]
   \centering
   \includegraphics[scale=0.05]{Images/ts.png} 
   \caption{Logo TypeScript}
   \label{fig:ts}
\end{figure}

TypeScript, développé par Microsoft, est un surensemble de 
JavaScript qui ajoute des types statiques, permettant de détecter 
les erreurs dès la phase de développement. Il compile en JavaScript 
standard et est compatible avec tous les navigateurs. 
TypeScript offre des fonctionnalités avancées comme les 
interfaces, les énumérations et les génériques, et bénéficie 
d'un excellent support des outils de développement, facilitant 
l'auto-complétion et la refactorisation. Il est idéal pour les 
applications à grande échelle où la qualité du code est importante.
\newline

\large
\textbf{HTML}
\begin{figure}[htbp]
   \centering
   \includegraphics[scale=0.2]{Images/html.png} 
   \caption{Logo HTML}
   \label{fig:html}
\end{figure}

HTML est un langage de balisage conçu pour représenter les pages
 web. C’est un langage permettant d’écrire de l’hypertexte, 
 d’où son nom. HTML permet également de structurer sémantiquement 
 et logiquement et de mettre en forme le contenu des pages, 
 d’inclure des ressources multimédias dont des images, des 
 formulaires de saisie et des programmes informatiques.

 

\subsection{Frameworks}
\large
\textbf{React}
\begin{figure}[htbp]
   \centering
   \includegraphics[scale=0.1]{Images/react.png} 
   \caption{Logo React}
   \label{fig:react}
\end{figure}

React est une bibliothèque JavaScript frontale open source 
permettant de créer des interfaces utilisateur ou des composants 
d’interface utilisateur. Il est maintenu par Fa- cebook et une 
communauté de développeurs individuels et d’entreprises. React 
peut être utilisé comme base dans le dé- veloppement d’applications 
monopages ou mobiles.
\newline

\large
\textbf{Tailwind}
\begin{figure}[htbp]
   \centering
   \includegraphics[scale=0.6]{Images/tailwind.png} 
   \caption{Logo Tailwind}
   \label{fig:tailwind}
\end{figure}

Tailwind est un framework CSS qui fournit un catalogue complet de 
classes et d’outils CSS qui vous permet de commencer facilement 
à styliser votre site Web ou votre application.
\newline

\large
\textbf{Cypress}
\begin{figure}[htbp]
   \centering
   \includegraphics[scale=0.6]{Images/cy.jpg} 
   \caption{Logo Cypress}
   \label{fig:4D}
\end{figure}

Cypress est un framework de test open-source populaire utilisé 
pour automatiser les tests d’applications web. Il permet aux 
développeurs d’écrire des tests end-to-end (de bout en bout) pour 
vérifier le bon fonctionnement des applications web.

% travail realise
\section{travail résalisé}
\subsection{Authentification}

\subsection{Espace Candidat}
\subsubsection{Modifier le profile}

\subsubsection{Consulter les offres disponibles}

\subsubsection{Consulter les details d'une offre}


\subsubsection{Postuler à une offre}


\subsubsection{Passer un test}

\subsubsection{Consulter le calendrier}


\subsection{Espace Recruteur}
\subsubsection{Consulter le dashboard}

\subsubsection{Publier une offre}

\subsubsection{Programmer un entretien}


\subsection{Espace Administrateur}

\subsubsection{Consulter le dashboard}

\subsubsection{Ajouter un recruteur}

\section{Test et Validation}
\subsection{Tests unitaires}
Jest

\subsection{Tests de bout en bout}
Cypress

