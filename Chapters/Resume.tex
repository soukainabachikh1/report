\chapter*{Résumé}
\addcontentsline{toc}{chapter}{Résumé}

% \hspace{0.6cm}Le processus de recrutement est un élément crucial pour toute entreprise souhaitant maintenir sa compétitivité sur le marché du travail. Dans le cadre de ce projet de fin d'études réalisé à l'Institut National des Postes et Télécommunications (INPT) en génie logiciel, nous avons entrepris la conception et le développement d'une plateforme de gestion des recrutements pour l'entreprise 4D. Cette plateforme vise à rationaliser et à automatiser les différentes étapes du processus de recrutement, du sourcing des candidats à l'intégration dans l'entreprise.
% \newline

% Le projet s'est articulé autour de plusieurs phases clés, comprenant une analyse approfondie des besoins de l'entreprise, la conception d'une architecture logicielle adaptée, le développement de fonctionnalités sur mesure et la mise en œuvre d'outils de suivi et d'évaluation. En utilisant les meilleures pratiques de développement logiciel et les technologies les plus récentes, nous avons pu créer une plateforme robuste et évolutive, capable de répondre aux exigences spécifiques de l'entreprise 4D.
% \newline

% Les fonctionnalités principales de la plateforme comprennent la gestion des offres d'emploi, la diffusion automatique sur différents canaux de recrutement, la gestion des candidatures, le suivi des entretiens, la collaboration entre les différents intervenants du processus de recrutement et la génération de rapports analytiques pour évaluer l'efficacité des stratégies de recrutement mises en œuvre.
% \newline

% Ce projet a permis de mettre en pratique les connaissances théoriques acquises au cours de notre formation en génie logiciel à l'INPT et de développer des compétences essentielles en matière de conception et de développement de solutions logicielles. Il constitue également une contribution significative à l'optimisation des processus de recrutement de l'entreprise 4D, en permettant une gestion plus efficace et plus transparente des ressources humaines.
% \newline
\begin{spacing}{1.25}

Ce rapport présente le travail effectué lors de mon stage de fin d’études 
chez 4D Logiciels Maroc, axé sur le développement 
d’une plateforme de gestion de recrutement. L'objectif principal 
de ce projet est de simplifier et d'automatiser le processus de recrutement,
réduisant ainsi les délais de traitement des candidatures et les efforts effectués par les recruteurs. 
Il y parviendra en intégrant des fonctionnalités principales, notamment la 
gestion des offres d'emploi, le tri automatique des CVs, l'évaluation des compétences des candidats via des tests d'évaluation 
et la planification des entretiens dans un seul système 
cohérent.
\newline

La méthodologie agile 
a été adoptée, avec des itérations rapides et des ajustements 
continus en fonction des besoins, compte tenu de l'absence 
d'un cahier des charges bien défini. 
Pour mener à bien ce projet, nous avons suivi plusieurs étapes. 
Tout d’abord, une analyse approfondie des besoins a été réalisée 
afin d'identifier les exigences spécifiques des candidats, 
des recruteurs et des administrateurs. Ensuite, la modéli-
sation 
du système a été effectuée à l’aide des diagrammes UML et BPMN. 
Enfin, le développement des fonctionnalités a été entrepris en se 
basant principalement sur 4D en backend et React en frontend, tout en 
intégrant des tests pour assurer la qualité et la fiabilité du produit final.

\end{spacing}

\vspace{1cm}
\noindent\rule[2pt]{\textwidth}{0.5pt}

{\textbf{Mot clés :} 
recrutement,
tri automatique des CVs,
évaluation des compétences, planification des entretiens
, UML, BPMN, 4D, React}
\\
% \noindent\rule[2pt]{\textwidth}{0.5pt}