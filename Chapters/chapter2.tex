% Define the page style
\fancypagestyle{chapterstyle}{
   \fancyhead[L]{\nouppercase{\rightmark}}
   \fancyhead[R]{Projet de fin d'etudes 2023-2024}
   \fancyfoot[C]{\vspace{20pt}\thepage} % Adjust the vertical space here
   \setlength{\headheight}{20pt}
   \setlength{\footskip}{30pt} % Adjust the value as needed
}


\chapter{Analyse et Spécifications des besoins}
\pagestyle{chapterstyle}


\newpage
\vspace{1cm}
%---------Introduction----------
\section{Introduction}

%-------------Spécifications des exigences-----------
\section{Spécifications des exigences}

\subsection{Analyse de l'existant}

a. Etude de l'existant :

Dans le cadre de processus de recrutement de 4D, avant la mise en place de notre projet de digitalisation du processus de recrutement, plusieurs étapes clés sont suivies pour identifier, évaluer et sélectionner les meilleurs candidats pour nos postes vacants. Ci-dessous, nous décrivons en détail les principales étapes, allant du dépôt de l'offre d'emploi sur LinkedIn à la planification des entretiens, qui constituent notre processus de recrutement avant toute digitalisation.

· Dépôt de l'Offre d'Emploi sur LinkedIn :

Les responsables RH ou les recruteurs préparent des annonces d'emploi complètes sur la plateforme LinkedIn. Ces annonces comprennent des détails tels que le titre du poste, les responsabilités, les compétences requises, le type de contrat, la localisation, le niveau d'expérience souhaité, les avantages offerts, etc. Une attention particulière est portée à la structuration et à la clarté de ces annonces. Après leur création, les annonces sont publiées sur la page LinkedIn de

L’entreprise ou dans des groupes professionnels appropriés pour atteindre un public ciblé de candidats potentiels.

· Réception et Tri des CVs :

Une fois les annonces publiées, les recruteurs commencent à recevoir les CVs des candidats. Ces CVs sont envoyés par e-mail en réponse aux annonces sur LinkedIn. Les recruteurs examinent attentivement chaque CV pour évaluer la correspondance avec les critères définis dans l'annonce. Les compétences techniques, l'expérience professionnelle pertinente, l'éducation, les réalisations antérieures et autres informations clés sont prises en compte lors du tri manuel des CVs. Les candidats dont les profils correspondent le mieux aux exigences de l'offre sont présélectionnés pour la suite du processus.

· Planification des Entretiens :

Après avoir présélectionné les candidats pour un entretien, ces derniers reçoivent e-mail d'invitation détaillant les informations sur l'entretien à venir, telles que le format (téléphonique, vidéoconférence, en personne, etc.). Dans cet e-mail, les candidats sont invités à choisir leur propre date et heure préférées pour l'entretien en utilisant un lien vers Google Calendar.

Une fois que le candidat a sélectionné une plage horaire, le système envoie automatiquement une confirmation aux recruteurs, les notifiant de la disponibilité choisie par le candidat et bloquant cette plage horaire dans le calendrier partagé pour éviter les conflits de planification.

Les recruteurs reçoivent une notification instantanée de la disponibilité choisie par le candidat et peuvent ainsi finaliser la planification de l'entretien. Des rappels sont également programmés pour être envoyés aux candidats et aux recruteurs quelques jours avant la date prévue de l'entretien afin de garantir une participation ponctuelle et bien préparée de toutes les parties concernées.

Chapitre 2 : Etude détaillée du projet

b. Limitation de l’existant

Malgré ses bénéfices, tels que la diffusion efficace des offres d'emploi sur LinkedIn et la facilité de gestion des entretiens grâce à Google Calendar, le processus de recrutement actuel comporte également plusieurs limites importantes. Tout d'abord, le processus de tri manuel des CV par les recruteurs est une tâche complexe et susceptible d'être sujette à des erreurs humaines. Cela nécessite un investissement de temps important pour évaluer chaque CV en fonction des critères de l'annonce, ce qui peut entraîner des retards dans le processus de sélection. En outre, cette méthode manuelle ne permet pas d'analyser en détail les compétences et l'expérience des candidats, ce qui restreint la capacité à repérer les meilleurs talents. La planification des entretiens demeure également manuelle et peut entraîner des problèmes de coordination et de communication, malgré l'utilisation de Google Calendar. La gestion des disponibilités des candidats et des recruteurs n'est pas automatisée, ce qui peut causer des retards et des erreurs dans la planification des entretiens. Ensuite, la gestion des candidatures et la production de rapports détaillés sur le processus de recrutement demandent actuellement une grande quantité de travail manuel et peuvent poser des problèmes pour évaluer l'efficacité des stratégies de recrutement. En outre, il est important de noter que le processus actuel ne permet pas aux candidats de suivre efficacement l'évolution de leur candidature, ce qui peut entraîner une expérience utilisateur insatisfaisante. De plus, en utilisant des plateformes externes telles que LinkedIn et Google Calendar, l'entreprise n'a pas sa propre solution interne personnalisée pour gérer le processus de recrutement, ce qui peut entraîner des contraintes en termes de personnalisation, de confidentialité des données et de contrôle sur le flux de recrutement.

3. Analyse et identification

\subsection{Identifications des acteurs }

\subsection{Exigences fonctionnels}

\subsection{Exigences non fonctionnels}

%----------Identification des cas d'utilisation ----------