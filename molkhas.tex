\chapter*{\<ملخص>}
\addcontentsline{toc}{chapter}{\<ملخص>}

\hspace{0.6cm}
\begin{arabtex}
     العمل. في إطار هذا المشروع الذي تم تنفيذه في مؤسسة 4D حيث أجريت تدريبك العملي، قمت بتصميم وتطوير منصة لإدارة التوظيف. تهدف هذه المنصة إلى تبسيط وتأمين مراحل عملية التوظيف المختلفة، بدءًا من تجنيد المرشحين وصولًا إلى التوظيف في الشركة.

شمل المشروع عدة مراحل رئيسية، بما في ذلك تحليل شامل لاحتياجات الشركة، وتصميم هندسة برمجية مناسبة، وتطوير ميزات مخصصة، وتنفيذ أدوات لتتبع وتقييم العمليات. باستخدام أفضل الممارسات في تطوير البرمجيات وأحدث التقنيات، تمكنت من إنشاء منصة قوية وقابلة للتوسيع قادرة على تلبية المتطلبات المحددة لمؤسسة 4D.

تتضمن الميزات الرئيسية للمنصة إدارة الوظائف، والتوزيع التلقائي عبر مختلف قنوات التوظيف، وإدارة المرشحين، وتتبع المقابلات، والتعاون بين أطراف العملية التوظيفية، وإنشاء تقارير تحليلية لتقييم فعالية استراتيجيات التوظيف المطبقة.

ساهم هذا المشروع في تطبيق المعرفة النظرية المكتسبة خلال التدريب في هندسة البرمجيات وتطوير مهارات أساسية في تصميم وتطوير حلول برمجية. كما يمثل مساهمة هامة في تحسين عمليات التوظيف لمؤسسة 4D، مما يمكن من إدارة الموارد البشرية بشكل أكثر كفاءة وشفافية.

\end{arabtex}
